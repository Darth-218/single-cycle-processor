\documentclass[conference, twocolumn]{IEEEtran}

\usepackage{graphicx}
\usepackage{amsmath}
\usepackage{amssymb}
\usepackage{cite}
\usepackage{float}
\usepackage{url}

\title{Design and Implementation of a Single-Cycle RISC-V Processor}

\author{
\IEEEauthorblockN{
Ahmed Mohamed Abdelmaboud,
Yahia Hany Gaber,
Omar Hesham Desouky,
Mahmoud Khaled,
Mazen Ahmed
}
\IEEEauthorblockA{}
}

\begin{document}

\maketitle

\begin{abstract}
This paper presents the design and implementation of a 64-bit single-cycle RISC-V processor. The processor executes each instruction in a single clock cycle and supports arithmetic and logic operations, memory access instructions, and conditional branching. The design follows the standard single-cycle RISC-V datapath, with a merged Main Control Unit and ALU Control Unit. All modules were implemented using Verilog HDL and verified through simulation.
\end{abstract}

\begin{IEEEkeywords}
RISC-V, Single-Cycle Processor, Computer Architecture, Verilog, CPU Design
\end{IEEEkeywords}

\section{Introduction}
The purpose of this project is to design and implement a single-cycle RISC-V processor as part of a Computer Architecture course. In a single-cycle architecture, every instruction is fetched, decoded, executed, and completed within one clock cycle. While this approach is not optimized for performance, it provides a clear and educational view of processor datapath and control design.

The processor supports R-type arithmetic and logic instructions, I-type arithmetic instructions, memory access operations, and conditional branching.

\section{Supported Instruction Set}
The processor supports the following RISC-V instruction categories:

\subsection{R-Type Instructions}
\begin{itemize}
\item add, sub
\item and, or, xor
\item sll, srl
\end{itemize}

\subsection{I-Type Arithmetic Instructions}
\begin{itemize}
\item addi, andi, ori, xori
\end{itemize}

\subsection{Shift Instructions}
\begin{itemize}
\item Immediate shifts: slli, srli
\item Register-based shifts: sll, srl
\end{itemize}

\subsection{Memory Access Instructions}
\begin{itemize}
\item Load doubleword (ld)
\item Store doubleword (sd)
\end{itemize}

\subsection{Branch Instruction}
\begin{itemize}
\item beq
\end{itemize}

\section{Datapath Design}
The processor datapath is based on the standard single-cycle RISC-V architecture. It includes the Program Counter (PC), Instruction Memory, Register File, Immediate Generator, Arithmetic Logic Unit (ALU), Data Memory, Control Unit, and branch logic.

\begin{figure}[H]
\centering
\includegraphics[width=0.8\textwidth]{datapath.png}
\caption{Single-Cycle RISC-V Datapath}
\label{fig:datapath}
\end{figure}

\section{Design Methodology}
A modular design approach was followed. Each processor component was implemented as a separate Verilog module and verified using an individual testbench. After verification, all modules were integrated into a top-level CPU module.

The ALU Control logic was merged with the Main Control Unit to simplify control signal generation.

\section{Control Unit}
The Control Unit decodes the instruction opcode, funct3, and funct7 fields and generates all necessary control signals, including RegWrite, MemRead, MemWrite, ALUSrc, MemToReg, Branch, and ALUControl.

\begin{table}[H]
\centering
\caption{Full Control Unit Truth Table}
\scriptsize
\resizebox{\columnwidth}{!}{%
\begin{tabular}{|l|c|c|c|c|c|c|c|c|c|c|c|c|}
\hline
Instruction & Opcode & funct3 & funct7 & RegWrite & MemRead & MemWrite & MemToReg & Branch & ALUSrc & ALUctl & pc\_src & imm\_type \\ \hline
ADD    & 0110011 & 000 & 0000000 & 1 & 0 & 0 & 0 & 0 & 0 & 0010 (ADD) & 00 & - \\
SUB    & 0110011 & 000 & 0100000 & 1 & 0 & 0 & 0 & 0 & 0 & 0110 (SUB) & 00 & - \\
AND    & 0110011 & 111 & 0000000 & 1 & 0 & 0 & 0 & 0 & 0 & 0000 (AND) & 00 & - \\
OR     & 0110011 & 110 & 0000000 & 1 & 0 & 0 & 0 & 0 & 0 & 0001 (OR) & 00 & - \\
XOR    & 0110011 & 100 & 0000000 & 1 & 0 & 0 & 0 & 0 & 0 & 0011 (XOR) & 00 & - \\
SLL    & 0110011 & 001 & 0000000 & 1 & 0 & 0 & 0 & 0 & 0 & 0100 (SLL) & 00 & - \\
SRL    & 0110011 & 101 & 0000000 & 1 & 0 & 0 & 0 & 0 & 0 & 0101 (SRL) & 00 & - \\
SRA    & 0110011 & 101 & 0100000 & 1 & 0 & 0 & 0 & 0 & 0 & 1001 (SRA) & 00 & - \\
SLT    & 0110011 & 010 & 0000000 & 1 & 0 & 0 & 0 & 0 & 0 & 0111 (SLT) & 00 & - \\
SLTU   & 0110011 & 011 & 0000000 & 1 & 0 & 0 & 0 & 0 & 0 & 1000 (SLTU) & 00 & - \\
ADDI   & 0010011 & 000 & -       & 1 & 0 & 0 & 0 & 0 & 1 & 0010 (ADD) & 00 & 000 \\
ANDI   & 0010011 & 111 & -       & 1 & 0 & 0 & 0 & 0 & 1 & 0000 (AND) & 00 & 000 \\
ORI    & 0010011 & 110 & -       & 1 & 0 & 0 & 0 & 0 & 1 & 0001 (OR)  & 00 & 000 \\
XORI   & 0010011 & 100 & -       & 1 & 0 & 0 & 0 & 0 & 1 & 0011 (XOR) & 00 & 000 \\
SLLI   & 0010011 & 001 & 0000000 & 1 & 0 & 0 & 0 & 0 & 1 & 0100 (SLL) & 00 & 000 \\
SRLI   & 0010011 & 101 & 0000000 & 1 & 0 & 0 & 0 & 0 & 1 & 0101 (SRL) & 00 & 000 \\
SRAI   & 0010011 & 101 & 0100000 & 1 & 0 & 0 & 0 & 0 & 1 & 1001 (SRA) & 00 & 000 \\
SLTI   & 0010011 & 010 & -       & 1 & 0 & 0 & 0 & 0 & 1 & 0111 (SLT) & 00 & 000 \\
SLTIU  & 0010011 & 011 & -       & 1 & 0 & 0 & 0 & 0 & 1 & 1000 (SLTU)& 00 & 000 \\
LW     & 0000011 & -   & -       & 1 & 1 & 0 & 1 & 0 & 1 & 0010 (ADD) & 00 & 000 \\
SW     & 0100011 & -   & -       & 0 & 0 & 1 & 0 & 0 & 1 & 0010 (ADD) & 00 & 001 \\
BEQ    & 1100011 & 000 & -       & 0 & 0 & 0 & 0 & 1 & 0 & 0110 (SUB) & 01 & 010 \\
JAL    & 1101111 & -   & -       & 1 & 0 & 0 & 0 & 0 & 0 & -           & 10 & 100 \\
JALR   & 1100111 & -   & -       & 1 & 0 & 0 & 0 & 0 & 1 & 0010 (ADD) & 10 & 000 \\
LUI    & 0110111 & -   & -       & 1 & 0 & 0 & 0 & 0 & 1 & 0010 (ADD) & 00 & 011 \\
AUIPC  & 0010111 & -   & -       & 1 & 0 & 0 & 0 & 0 & 1 & 0010 (ADD) & 00 & 011 \\
\hline
\end{tabular}%
}
\end{table}

\section{Test Program}
To verify correct functionality of the single-cycle RISC-V processor, a custom assembly test program was written to exercise arithmetic, logical, shift, memory, and branch instructions.

\begin{verbatim}
addi x0,  x0,  7
addi x11, x0, 196
addi x3,  x0,  4
add  x2,  x11, x3
and  x24, x2,  x11
srl  x27, x24, x3
sub  x22, x27, x27
sd   x27, 0(x0)
ld   x19, 0(x0)
beq  x19, x27, L1
addi x13, x27, 7
beq  x0,  x0,  END
L1:  addi x13, x0,  17
END: addi x0,  x0,  0
\end{verbatim}

\subsection{Expected Results}
\begin{itemize}
\item Register x0 = 0 (hardwired to zero)
\item Register x27 = 12
\item Register x22 = 0
\item Register x19 = 12
\item Register x13 = 17
\item Memory location mem[0] = 12
\end{itemize}

\begin{figure}[H]
    \centering
    \includegraphics[width=0.9\linewidth]{}
\begin{figure}
        \centering
        \includegraphics[width=1\linewidth]{vivado sim waveform.jpeg}
        \caption{Vivado simulation waveform
        the expected final states (registers and memory) and the value of the pc for the test program.}
        \label{fig:placeholder}
    \end{figure}
\end{figure}


\section{Module Implementation}
\subsection{Arithmetic Logic Unit}
The ALU supports arithmetic, logical, and shift operations required by R-type and I-type instructions.

\subsection{Register File}
The register file contains 32 general-purpose 64-bit registers. Register x0 is permanently hardwired to zero. The module supports two asynchronous read ports and one synchronous write port.

\subsection{Immediate Generator}
The Immediate Generator extracts and sign-extends immediates for I-type, S-type, B-type, and shift-immediate instructions.

\subsection{Program Counter and Branch Logic}
The PC updates every clock cycle. Branch target addresses are calculated using the sign-extended immediate shifted left by one bit. Branches are taken when the Branch signal is asserted and the ALU zero flag is set.

\subsection{Memory Units}
Instruction memory is preloaded with RISC-V machine code. Data memory supports 64-bit load and store operations.

\section{Simulation Results}
Simulation was performed using Verilog testbenches. The results verified correct PC updates, instruction execution, register writes, ALU outputs, memory operations, and branch behavior. Waveforms demonstrate correct processor functionality.

\section{Conclusion}
This project successfully demonstrates the design and implementation of a 64-bit single-cycle RISC-V processor. All required instruction categories were supported and verified through simulation. The design provides a strong foundation for understanding more advanced processor architectures.

\begin{thebibliography}{1}
\bibitem{riscv}
RISC-V Foundation, \emph{The RISC-V Instruction Set Manual}, Volume I.

\bibitem{patterson}
D. A. Patterson and J. L. Hennessy, \emph{Computer Organization and Design: RISC-V Edition}.
\end{thebibliography}

\end{document}
\
