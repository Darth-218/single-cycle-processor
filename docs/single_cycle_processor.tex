\documentclass[10pt]{beamer}
\usepackage{graphicx}
\usepackage{booktabs}
\usepackage{tabularx}
\usepackage{listings}
\usepackage{xcolor}
\usepackage{hyperref}
\usepackage{adjustbox}
\usepackage{multirow}
\usepackage{rotating}
\usepackage{makecell}
\usepackage{caption}

\usetheme{Madrid}
\usecolortheme{whale}

% Fix for author lines in PDF metadata
\title{Design and Implementation of a Single-Cycle RISC-V Processor}
\author{Ahmed Mohamed Abdelmaboud \and Yahia Hany Gaber \and Omar Hesham Desouky \and Mahmoud Khaled \and Mazen Ahmed}
\institute{Computer Architecture Project}
\date{}

\begin{document}

% ========== TITLE SLIDE ==========
\begin{frame}
    \titlepage
\end{frame}

% ========== ABSTRACT ==========
\begin{frame}{Abstract}
    \begin{block}{Objective}
        Design and implementation of a 64-bit single-cycle RISC-V processor that executes each instruction in one clock cycle.
    \end{block}
    
    \begin{itemize}
        \item Supports arithmetic, logic, memory access, and branching
        \item Follows standard single-cycle RISC-V datapath
        \item Merged Main Control Unit and ALU Control Unit
        \item Implemented in Verilog HDL with simulation verification
    \end{itemize}
    
    \vspace{0.5cm}
    \textbf{Index Terms:} RISC-V, Single-Cycle Processor, Computer Architecture, Verilog, CPU Design
\end{frame}

% ========== INTRODUCTION ==========
\begin{frame}{Introduction}
    \begin{block}{Project Goal}
        Educational implementation of a single-cycle RISC-V processor for a Computer Architecture course.
    \end{block}
    
    \begin{itemize}
        \item \textbf{Single-cycle architecture:} Each instruction completes in one clock cycle
        \item Not performance-optimized, but educationally valuable
        \item Clear demonstration of datapath and control design
        \item Supports:
        \begin{itemize}
            \item R-type and I-type arithmetic/logic instructions
            \item Memory access (load/store)
            \item Conditional branching
        \end{itemize}
    \end{itemize}
\end{frame}

% ========== INSTRUCTION SET ==========
\begin{frame}{Supported Instruction Set}
    \begin{columns}[T]
        \begin{column}{0.48\textwidth}
            \begin{block}{R-Type Instructions}
                \begin{itemize}
                    \item add, sub
                    \item and, or, xor
                    \item sll, srl
                \end{itemize}
            \end{block}
            
            \begin{block}{I-Type Arithmetic Instructions}
                \begin{itemize}
                    \item addi, andi, ori, xori
                \end{itemize}
            \end{block}
        \end{column}
        
        \begin{column}{0.48\textwidth}
            \begin{block}{Shift Instructions}
                \begin{itemize}
                    \item Immediate shifts: slli, srli
                    \item Register-based: sll, srl
                \end{itemize}
            \end{block}
            
            \begin{block}{Memory Instructions}
                \begin{itemize}
                    \item ld (load doubleword)
                    \item sd (store doubleword)
                \end{itemize}
            \end{block}
            
            \begin{block}{Branch Instruction}
                \begin{itemize}
                    \item beq (branch if equal)
                \end{itemize}
            \end{block}
        \end{column}
    \end{columns}
\end{frame}

% ========== DATAPATH DESIGN ==========
\begin{frame}{Datapath Design}
    \begin{center}
        % Use one of these lines based on which file you have:
        \includegraphics[width=0.6\textwidth]{Screenshot 2025-12-29 at 01-58-00 311 - Project-V2-1.pdf.png} % Option 1: Simple filename
        % \includegraphics[width=0.8\textwidth]{Screenshot 2025-12-29 at 01-58-00 311 - Project-V2-1.pdf.png} % Option 2: Original filename
        
        \vspace{0.3cm}
        \footnotesize\textbf{Fig. 1:} Single-Cycle RISC-V Datapath
    \end{center}
    
\end{frame}


\begin{frame}{Dataapath Components}
    \begin{itemize}
        \item \textbf{Program Counter (PC):} Holds address of next instruction
        \item \textbf{Instruction Memory:} Stores program instructions
        \item \textbf{Register File:} 32 registers for data storage and manipulation
        \item \textbf{ALU:} Performs arithmetic and logic operations
        \item \textbf{Data Memory:} For load/store operations
        \item \textbf{Sign/Zero Extender:} Extends immediate values to 64 bits
        \item \textbf{Multiplexers (MUXes):} Select between different data sources
    \end{itemize}
\end{frame}

% ========== SIMPLIFIED CONTROL TABLE (FITS BETTER) ==========
\begin{frame}{Control Unit Signals (Selected Instructions)}
    \scriptsize
    \centering
    \begin{tabular}{|l|c|c|c|c|c|c|c|}
        \hline
        \textbf{Instruction} & \textbf{RegWrite} & \textbf{MemRead} & \textbf{MemWrite} & \textbf{MemToReg} & \textbf{Branch} & \textbf{ALUSrc} & \textbf{ALUOp} \\
        \hline
        R-type (add, sub, etc.) & 1 & 0 & 0 & 0 & 0 & 0 & 10 \\
        \hline
        addi, andi, etc. & 1 & 0 & 0 & 0 & 0 & 1 & 00 \\
        \hline
        ld & 1 & 1 & 0 & 1 & 0 & 1 & 00 \\
        \hline
        sd & 0 & 0 & 1 & X & 0 & 1 & 00 \\
        \hline
        beq & 0 & 0 & 0 & X & 1 & 0 & 01 \\
        \hline
    \end{tabular}
    
    \vspace{0.5cm}
    
    \begin{tabular}{|l|c|c|c|}
        \hline
        \textbf{ALUOp} & \textbf{funct3} & \textbf{funct7} & \textbf{ALU Operation} \\
        \hline
        00 & XXX & XXXXXXX & ADD (for loads/stores) \\
        \hline
        01 & XXX & XXXXXXX & SUB (for branches) \\
        \hline
        10 & 000 & 0000000 & ADD \\
        \hline
        10 & 000 & 0100000 & SUB \\
        \hline
        10 & 111 & 0000000 & AND \\
        \hline
        10 & 110 & 0000000 & OR \\
        \hline
        10 & 100 & 0000000 & XOR \\
        \hline
    \end{tabular}
\end{frame}

% ========== TEST PROGRAM ==========
\begin{frame}{Test Program}
    \begin{exampleblock}{Assembly Test Program}
        \texttt{
        addi x0, x0, 7 \\
        addi x11, x0, 196 \\
        addi x3, x0, 4 \\
        add x2, x11, x3 \\
        and x24, x2, x11 \\
        srl x27, x24, x3 \\
        sub x22, x27, x27 \\
        sd x27, 0(x0) \\
        ld x19, 0(x0) \\
        beq x19, x27, L1 \\
        addi x13, x27, 7 \\
        beq x0, x0, END \\
        L1: addi x13, x0, 17 \\
        END: addi x0, x0, 0}
    \end{exampleblock}
\end{frame}

% ========== EXPECTED RESULTS ==========
\begin{frame}{Expected Results}
    \begin{columns}[T]
        \begin{column}{0.48\textwidth}
            \begin{block}{Register Values}
                \begin{itemize}
                    \item Register x0 = 0 (hardwired to zero)
                    \item Register x27 = 12
                    \item Register x22 = 0
                    \item Register x19 = 12
                    \item Register x13 = 17
                \end{itemize}
            \end{block}
        \end{column}
        
        \begin{column}{0.48\textwidth}
            \begin{block}{Memory State}
                \begin{itemize}
                    \item Memory location mem[0] = 12
                \end{itemize}
            \end{block}
            
            \begin{alertblock}{Verification Method}
                \begin{itemize}
                    \item Values derived analytically
                    \item Used as reference for simulation
                    \item Tests all instruction types
                \end{itemize}
            \end{alertblock}
        \end{column}
    \end{columns}
\end{frame}

% ========== SIMULATION RESULTS ==========
\begin{frame}{Simulation Results}
    \begin{center}
        % Use one of these lines based on which file you have:
        \includegraphics[width=0.8\textwidth]{WhatsApp Image 2025-12-29 at 3.07.36 AM.jpeg} % Option 1: Simple filename
        % \includegraphics[width=0.8\textwidth]{WhatsApp Image 2025-12-29 at 3.07.36 AM.jpeg} % Option 2: Original filename
        
        \vspace{0.3cm}
        \footnotesize\textbf{Fig. 2:} Vivado Simulation Waveform
    \end{center}
    
    \begin{block}{Verified Components}
        \begin{itemize}
            \item Correct PC updates and control flow
            \item Instruction execution and decoding
            \item Register file operations
            \item ALU operations
            \item Memory access operations
            \item Branch behavior
        \end{itemize}
    \end{block}
\end{frame}

% ========== CONCLUSION ==========
\begin{frame}{Conclusion}
    \begin{block}{Project Achievements}
        \begin{itemize}
            \item Successfully designed and implemented 64-bit single-cycle RISC-V processor
            \item Comprehensive control unit with truth table
            \item All instruction categories supported and verified
            \item Modular Verilog implementation
            \item Complete simulation verification with Vivado
        \end{itemize}
    \end{block}
    
    \begin{block}{Educational Value}
        \begin{itemize}
            \item Provides clear understanding of processor datapath
            \item Demonstrates control unit design principles
            \item Hands-on experience with CPU implementation
            \item Foundation for advanced processor architectures
        \end{itemize}
    \end{block}
\end{frame}

% ========== REFERENCES ==========
\begin{frame}{References}
    \footnotesize
       \vspace{0.5cm}
    \begin{center}
        \Large Thank You \\
        \normalsize Questions?
    \end{center}
\end{frame}

\end{document}
