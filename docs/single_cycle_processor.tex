\documentclass[10pt]{beamer}
\usepackage{graphicx}
\usepackage{booktabs}
\usepackage{tabularx}
\usepackage{listings}
\usepackage{xcolor}
\usepackage{hyperref}
\usepackage{adjustbox}
\usepackage{multirow}
\usepackage{rotating}
\usepackage{makecell}
\usepackage{caption}

\usetheme{Madrid}
\usecolortheme{whale}

\title{Design and Implementation of a Single-Cycle RISC-V Processor}
\author{Ahmed Mohamed Abdelmaboud \\ Yahia Hany Gaber \\ Omar Hesham Desouky \\ Mahmoud Khaled \\ Mazen Ahmed}
\institute{Computer Architecture Project}
\date{}

\begin{document}

% ========== TITLE SLIDE ==========
\begin{frame}
    \titlepage
\end{frame}

% ========== ABSTRACT ==========
\begin{frame}{Abstract}
    \begin{block}{Objective}
        Design and implementation of a 64-bit single-cycle RISC-V processor that executes each instruction in one clock cycle.
    \end{block}
    
    \begin{itemize}
        \item Supports arithmetic, logic, memory access, and branching
        \item Follows standard single-cycle RISC-V datapath
        \item Merged Main Control Unit and ALU Control Unit
        \item Implemented in Verilog HDL with simulation verification
    \end{itemize}
    
    \vspace{0.5cm}
    \textbf{Index Terms:} RISC-V, Single-Cycle Processor, Computer Architecture, Verilog, CPU Design
\end{frame}

% ========== INTRODUCTION ==========
\begin{frame}{Introduction}
    \begin{block}{Project Goal}
        Educational implementation of a single-cycle RISC-V processor for a Computer Architecture course.
    \end{block}
    
    \begin{itemize}
        \item \textbf{Single-cycle architecture:} Each instruction completes in one clock cycle
        \item Not performance-optimized, but educationally valuable
        \item Clear demonstration of datapath and control design
        \item Supports:
        \begin{itemize}
            \item R-type and I-type arithmetic/logic instructions
            \item Memory access (load/store)
            \item Conditional branching
        \end{itemize}
    \end{itemize}
\end{frame}

% ========== INSTRUCTION SET ==========
\begin{frame}{Supported Instruction Set}
    \begin{columns}[T]
        \begin{column}{0.48\textwidth}
            \begin{block}{R-Type Instructions}
                \begin{itemize}
                    \item add, sub
                    \item and, or, xor
                    \item sll, srl
                \end{itemize}
            \end{block}
            
            \begin{block}{I-Type Arithmetic Instructions}
                \begin{itemize}
                    \item addi, andi, ori, xori
                \end{itemize}
            \end{block}
        \end{column}
        
        \begin{column}{0.48\textwidth}
            \begin{block}{Shift Instructions}
                \begin{itemize}
                    \item Immediate shifts: slli, srli
                    \item Register-based: sll, srl
                \end{itemize}
            \end{block}
            
            \begin{block}{Memory Instructions}
                \begin{itemize}
                    \item ld (load doubleword)
                    \item sd (store doubleword)
                \end{itemize}
            \end{block}
            
            \begin{block}{Branch Instruction}
                \begin{itemize}
                    \item beq (branch if equal)
                \end{itemize}
            \end{block}
        \end{column}
    \end{columns}
\end{frame}

% ========== DATAPATH DESIGN ==========
\begin{frame}{Datapath Design}
    \begin{center}
        \includegraphics[width=0.85\textwidth]{}
\begin{figure}
            \centering
            \includegraphics[width=0.5\linewidth]{Screenshot 2025-12-29 at 01-58-00 311 - Project-V2-1.pdf.png}
            \label{fig:placeholder}
        \end{figure}
                
        \vspace{0.3cm}
        \footnotesize\textbf{Fig. 1:} Single-Cycle RISC-V Datapath
    \end{center}
    
    \begin{block}{Components}
        \begin{itemize}
            \item Program Counter (PC)
            \item Instruction Memory
            \item Register File
            \item Immediate Generator
            \item Arithmetic Logic Unit (ALU)
            \item Data Memory
            \item Control Unit
            \item Branch Logic
        \end{itemize}
    \end{block}
\end{frame}

% ========== DESIGN METHODOLOGY ==========
\begin{frame}{Design Methodology}
    \begin{columns}[T]
        \begin{column}{0.48\textwidth}
            \begin{block}{Modular Approach}
                \begin{itemize}
                    \item Each component as separate Verilog module
                    \item Individual testbenches for verification
                    \item Integrated into top-level CPU module
                \end{itemize}
            \end{block}
        \end{column}
        
        \begin{column}{0.48\textwidth}
            \begin{block}{Control Unit Design}
                \begin{itemize}
                    \item Main Control Unit merged with ALU Control
                    \item Simplified control signal generation
                    \item Decodes: opcode, funct3, funct7
                \end{itemize}
            \end{block}
        \end{column}
    \end{columns}
\end{frame}

% ========== CONTROL UNIT TRUTH TABLE (PART 1) ==========
\begin{frame}{Full Control Unit Truth Table (Part 1/2)}
    \tiny
    \centering
    \begin{tabular}{|l|c|c|c|c|c|c|c|c|c|c|c|}
        \hline
        \textbf{Instruction} & \textbf{Opcode} & \textbf{funct3} & \textbf{funct7} & \textbf{RegWrite} & \textbf{MemRead} & \textbf{MemWrite} & \textbf{MemToReg} & \textbf{Branch} & \textbf{ALUSrc} & \textbf{ALUctl} & \textbf{pc\_src} \\
        \hline
        ADD & 0110011 & 000 & 0000000 & 1 & 0 & 0 & 0 & 0 & 0 & 0010 (ADD) & 00 \\
        SUB & 0110011 & 000 & 0100000 & 1 & 0 & 0 & 0 & 0 & 0 & 0110 (SUB) & 00 \\
        AND & 0110011 & 111 & 0000000 & 1 & 0 & 0 & 0 & 0 & 0 & 0000 (AND) & 00 \\
        OR & 0110011 & 110 & 0000000 & 1 & 0 & 0 & 0 & 0 & 0 & 0001 (OR) & 00 \\
        XOR & 0110011 & 100 & 0000000 & 1 & 0 & 0 & 0 & 0 & 0 & 0011 (XOR) & 00 \\
        SLL & 0110011 & 001 & 0000000 & 1 & 0 & 0 & 0 & 0 & 0 & 0100 (SLL) & 00 \\
        SRL & 0110011 & 101 & 0000000 & 1 & 0 & 0 & 0 & 0 & 0 & 0101 (SRL) & 00 \\
        SRA & 0110011 & 101 & 0100000 & 1 & 0 & 0 & 0 & 0 & 0 & 1001 (SRA) & 00 \\
        SLT & 0110011 & 010 & 0000000 & 1 & 0 & 0 & 0 & 0 & 0 & 0111 (SLT) & 00 \\
        SLTU & 0110011 & 011 & 0000000 & 1 & 0 & 0 & 0 & 0 & 0 & 1000 (SLTU) & 00 \\
        ADDI & 0010011 & 000 & - & 1 & 0 & 0 & 0 & 0 & 1 & 0010 (ADD) & 00 \\
        ANDI & 0010011 & 111 & - & 1 & 0 & 0 & 0 & 0 & 1 & 0000 (AND) & 00 \\
        ORI & 0010011 & 110 & - & 1 & 0 & 0 & 0 & 0 & 1 & 0001 (OR) & 00 \\
        \hline
    \end{tabular}
    
    \vspace{0.2cm}
    \textbf{Table I:} Full Control Unit Truth Table (Part 1)
\end{frame}

% ========== CONTROL UNIT TRUTH TABLE (PART 2) ==========
\begin{frame}{Full Control Unit Truth Table (Part 2/2)}
    \tiny
    \centering
    \begin{tabular}{|l|c|c|c|c|c|c|c|c|c|c|c|}
        \hline
        \textbf{Instruction} & \textbf{Opcode} & \textbf{funct3} & \textbf{funct7} & \textbf{RegWrite} & \textbf{MemRead} & \textbf{MemWrite} & \textbf{MemToReg} & \textbf{Branch} & \textbf{ALUSrc} & \textbf{ALUctl} & \textbf{pc\_src} \\
        \hline
        XORI & 0010011 & 100 & - & 1 & 0 & 0 & 0 & 0 & 1 & 0011 (XOR) & 00 \\
        SLLI & 0010011 & 001 & 0000000 & 1 & 0 & 0 & 0 & 0 & 1 & 0100 (SLL) & 00 \\
        SRLI & 0010011 & 101 & 0000000 & 1 & 0 & 0 & 0 & 0 & 1 & 0101 (SRL) & 00 \\
        SRAI & 0010011 & 101 & 0100000 & 1 & 0 & 0 & 0 & 0 & 1 & 1001 (SRA) & 00 \\
        SLTI & 0010011 & 010 & - & 1 & 0 & 0 & 0 & 0 & 1 & 0111 (SLT) & 00 \\
        SLTIU & 0010011 & 011 & - & 1 & 0 & 0 & 0 & 0 & 1 & 1000 (SLTU) & 00 \\
        LW & 0000011 & 010 & - & 1 & 1 & 0 & 1 & 0 & 1 & 0010 (ADD) & 00 \\
        SW & 0100011 & 010 & - & 0 & 0 & 1 & X & 0 & 1 & 0010 (ADD) & 00 \\
        BEQ & 1100011 & 000 & - & 0 & 0 & 0 & X & 1 & 0 & 0110 (SUB) & 01 \\
        JAL & 1101111 & - & - & 1 & 0 & 0 & 0 & 0 & 1 & 0010 (ADD) & 10 \\
        JALR & 1100111 & 000 & - & 1 & 0 & 0 & 0 & 0 & 1 & 0010 (ADD) & 10 \\
        LUI & 0110111 & - & - & 1 & 0 & 0 & 0 & 0 & 1 & 0010 (ADD) & 00 \\
        AUIPC & 0010111 & - & - & 1 & 0 & 0 & 0 & 0 & 1 & 0010 (ADD) & 00 \\
        \hline
    \end{tabular}
    
    \vspace{0.2cm}
    \textbf{Table I:} Full Control Unit Truth Table (Part 2)
    
    \begin{block}{Key Signals}
        \begin{itemize}
            \item \textbf{imm\_type:} 000 (I-type), 001 (S-type), 010 (B-type), 011 (U-type), 100 (J-type)
            \item \textbf{pc\_src:} 00 (PC+4), 01 (branch), 10 (jump)
            \item \textbf{X:} Don't care
        \end{itemize}
    \end{block}
\end{frame}

% ========== TEST PROGRAM ==========
\begin{frame}{Test Program}
    \begin{exampleblock}{Assembly Test Program}
        \texttt{
        addi x0, x0, 7 \\
        addi x11, x0, 196 \\
        addi x3, x0, 4 \\
        add x2, x11, x3 \\
        and x24, x2, x11 \\
        srl x27, x24, x3 \\
        sub x22, x27, x27 \\
        sd x27, 0(x0) \\
        ld x19, 0(x0) \\
        beq x19, x27, L1 \\
        addi x13, x27, 7 \\
        beq x0, x0, END \\
        L1: addi x13, x0, 17 \\
        END: addi x0, x0, 0}
    \end{exampleblock}
\end{frame}

% ========== EXPECTED RESULTS ==========
\begin{frame}{Expected Results}
    \begin{columns}[T]
        \begin{column}{0.48\textwidth}
            \begin{block}{Register Values}
                \begin{itemize}
                    \item Register x0 = 0 (hardwired to zero)
                    \item Register x27 = 12
                    \item Register x22 = 0
                    \item Register x19 = 12
                    \item Register x13 = 17
                \end{itemize}
            \end{block}
        \end{column}
        
        \begin{column}{0.48\textwidth}
            \begin{block}{Memory State}
                \begin{itemize}
                    \item Memory location mem[0] = 12
                \end{itemize}
            \end{block}
            
            \begin{alertblock}{Verification Method}
                \begin{itemize}
                    \item Values derived analytically
                    \item Used as reference for simulation
                    \item Tests all instruction types
                \end{itemize}
            \end{alertblock}
        \end{column}
    \end{columns}
\end{frame}

% ========== MODULE IMPLEMENTATION ==========
\begin{frame}{Module Implementation: Core Components}
    \begin{columns}[T]
        \begin{column}{0.48\textwidth}
            \begin{block}{Arithmetic Logic Unit (ALU)}
                \begin{itemize}
                    \item Supports arithmetic, logical, and shift operations
                    \item Required by R-type and I-type instructions
                    \item Controlled by ALUctl signals
                \end{itemize}
            \end{block}
            
            \begin{block}{Register File}
                \begin{itemize}
                    \item 32 general-purpose 64-bit registers
                    \item Register x0 hardwired to zero
                    \item 2 read ports, 1 write port
                \end{itemize}
            \end{block}
        \end{column}
        
        \begin{column}{0.48\textwidth}
            \begin{block}{Immediate Generator}
                \begin{itemize}
                    \item Extracts and sign-extends immediates
                    \item Supports I, S, B, U, J types
                    \item Generates imm\_type signal
                \end{itemize}
            \end{block}
            
            \begin{block}{Program Counter \& Branch}
                \begin{itemize}
                    \item PC updates every clock cycle
                    \item Branch target: imm << 1
                    \item Branch when Branch=1 \& Zero=1
                \end{itemize}
            \end{block}
        \end{column}
    \end{columns}
\end{frame}

\begin{frame}{Module Implementation: Memory \& Control}
    \begin{columns}[T]
        \begin{column}{0.48\textwidth}
            \begin{block}{Memory Units}
                \begin{itemize}
                    \item \textbf{Instruction Memory:} Preloaded with RISC-V machine code
                    \item \textbf{Data Memory:} 64-bit load/store operations
                    \item Controlled by MemRead/MemWrite signals
                \end{itemize}
            \end{block}
        \end{column}
        
        \begin{column}{0.48\textwidth}
            \begin{block}{Control Unit}
                \begin{itemize}
                    \item Decodes opcode, funct3, funct7
                    \item Generates all control signals (see Table I)
                    \item Determines pc\_src and imm\_type
                    \item Merged with ALU Control
                \end{itemize}
            \end{block}
        \end{column}
    \end{columns}
\end{frame}

% ========== SIMULATION RESULTS ==========
\begin{frame}{Simulation Results}
    \begin{block}{Simulation Methodology}
        \begin{itemize}
            \item Verilog testbenches for verification
            \item Vivado simulation environment
            \item Waveform analysis
        \end{itemize}
    \end{block}
    
    \begin{block}{Verified Components}
        \begin{itemize}
            \item Correct PC updates and control flow
            \item Instruction execution and decoding
            \item Register file operations
            \item ALU operations with all ALUctl codes
            \item Memory access operations
            \item Branch and jump behavior
        \end{itemize}
    \end{block}
\end{frame}

% ========== SIMULATION WAVEFORM ==========
\begin{frame}{Simulation Waveform}
    \begin{center}
        \includegraphics[width=0.9\textwidth]{}
\begin{figure}
            \centering
            \includegraphics[width=0.5\linewidth]{}
\begin{figure}
                \centering
                \includegraphics[width=0.75\linewidth]{}
\begin{figure}
                    \centering
                    \includegraphics[width=0.75\linewidth]{WhatsApp Image 2025-12-29 at 3.07.36 AM.jpeg}

                    \label{fig:placeholder}
                \end{figure}
                                \label{fig:placeholder}
            \end{figure}
        \label{fig:placeholder}
\end{figure}
                
        \vspace{0.3cm}
        \footnotesize\textbf{Fig. 2:} Vivado simulation waveform showing expected final states (registers and memory) and PC value for the test program
    \end{center}
    
    \begin{block}{Waveform Analysis}
        \begin{itemize}
            \item Shows correct register values (x27=12, x13=17, etc.)
            \item Demonstrates correct memory access (mem[0]=12)
            \item Verifies proper PC sequencing
            \item Confirms branch execution
        \end{itemize}
    \end{block}
\end{frame}

% ========== CONCLUSION ==========
\begin{frame}{Conclusion}
    \begin{block}{Project Achievements}
        \begin{itemize}
            \item Successfully designed and implemented 64-bit single-cycle RISC-V processor
            \item Comprehensive control unit with full truth table
            \item All instruction categories supported and verified
            \item Modular Verilog implementation
            \item Complete simulation verification with Vivado
        \end{itemize}
    \end{block}
    
    \begin{block}{Educational Value}
        \begin{itemize}
            \item Provides clear understanding of processor datapath
            \item Demonstrates control unit design principles
            \item Hands-on experience with CPU implementation
            \item Foundation for advanced processor architectures
        \end{itemize}
    \end{block}
\end{frame}

% ========== REFERENCES ==========
\begin{frame}{References}
    \footnotesize    
    \vspace{0.5cm}
    \begin{center}
        \Large Thank You \\
        \normalsize Questions?
    \end{center}
\end{frame}

\end{document}
